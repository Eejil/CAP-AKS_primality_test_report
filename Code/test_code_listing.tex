

\subsubsection*{Pseudocode:}

\begin{lstlisting}[style = PsC_Style, caption=Pseudocode]
for i in range(m-1):
    for j in range(i+1, m):
        [r,c] = np.where(M2 == M1[i,j])
        for k in range(len(r)):
            if M is None:
                M = np.copy(VT)
            else:
                M = np.concatenate((M, VT), 1)
            VT = np.zeros((n*m,1), int)
return M
\end{lstlisting}

\subsubsection*{Lean:}

\begin{lstlisting}[language=lean]
theorem funext {f₁ f₂ : ∀ (x : α), β x} (h : ∀ x, f₁ x = f₂ x) : f₁ = f₂ := by
  show extfunApp (Quotient.mk f₁) = extfunApp (Quotient.mk f₂)
  apply congrArg
  apply Quotient.sound
  exact h
\end{lstlisting}


\subsubsection*{Python:}

\begin{lstlisting}[
    language=python,
    style = PythonStyle,
    caption=Python example
    ]
import numpy as np
    
def incmatrix(genl1,genl2):
    m = len(genl1)
    n = len(genl2)
    M = None #to become the incidence matrix
    VT = np.zeros((n*m,1), int)  #dummy variable
    
    #compute the bitwise xor matrix
    M1 = bitxormatrix(genl1)
    M2 = np.triu(bitxormatrix(genl2),1) 

    for i in range(m-1):
        for j in range(i+1, m):
            [r,c] = np.where(M2 == M1[i,j])
            for k in range(len(r)):
                VT[(i)*n + r[k]] = 1;
                VT[(i)*n + c[k]] = 1;
                VT[(j)*n + r[k]] = 1;
                VT[(j)*n + c[k]] = 1;
                
                if M is None:
                    M = np.copy(VT)
                else:
                    M = np.concatenate((M, VT), 1)
                
                VT = np.zeros((n*m,1), int)
    
    return M
\end{lstlisting}


\subsubsection*{Sage:}

\begin{lstlisting}[language=Sage, style = SageStyle]
    sage: from sage.misc.citation import get_systems
    sage: get_systems('integrate(x^2, x)')
    ['ginac', 'Maxima']
    sage: R.<x,y,z> = QQ[]
    sage: I = R.ideal(x^2+y^2, z^2+y)
    sage: get_systems('I.primary_decomposition()')
    ['Singular']
\end{lstlisting}


\subsubsection*{MATLAB:}


\begin{lstlisting}[language=Matlab, style = MatlabStyle]
% matlab script sho.m
% finite difference scheme for simple harmonic oscillator

%%%%%% begin global parameters
SR = 44100;                         % sample rate (Hz)
f0 = 1000;                          % fundamental frequency (Hz)
TF = 1.0;                           % duration of simulation (s)
u0 = 0.3;                           % initial displacement
v0 = 0.0;                           % initial velocity

%%%%%% end global parameters

% check that stability condition is satisfied

if(SR<=pi*f0)
    error('Stability condition violated');
end
\end{lstlisting}


\clearpage