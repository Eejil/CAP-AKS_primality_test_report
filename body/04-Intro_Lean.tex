\documentclass[../main.tex]{subfiles}
\begin{document}

\subsection{Lean 4}


We formalized our proof using \textit{Lean 4}. 
Lean is an open-source, crowd-sourced, proof assistant %Lean is
based on dependant type theory and Calculus of Constructions. It is written in C++ developed and maintained by a diverse community of users, predominantly from mathematics and computer science.
Proof assistants, also called interactive theorem provers, are software tools that help mathematician formalize proofs rigorously by expressing every theorem in term of axioms and checking that these proofs are correct down to their logical foundation.  Proof assistant thus allow mathematicians to check their proof's correctness, but they also allow them to observe on which assumptions their proof and theorems do or don't rely on. 
% In Lean, many of those low level definitions are defined the the mathlib library; thus, proving mathematical statements in Lean requires making large use of it.
To prove mathematical statements in Lean requires making large use of the \textit{Mathlib} library. 
This library contains a substantial number of theorems, definitions, and proofs, as well as many tactics which makes theorem proving easier. 
It is also developed by the Lean community.

In this new age of Artificial Intelligence, we hope that growing library of Lean will find it's place.
As often happens in mathematics, what started as a fun project on a niche topic has already started to show it's use by giving suggestions on what step to use next in a proof, and could perhaps be used to guide researchers in the future. The Lean project was launched in 2013 by Leonardo de Moura, and already since 2018 it was grabbed the attention of leading mathematicians such as Terence Tao, Kevin Buzzard, Tomas Hals and Peter Scholze who have already been able to apply it to modern advances in mathematics. Such interest goes hand in hand with expanding the library and






\end{document}