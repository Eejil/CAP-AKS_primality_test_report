\documentclass[../Main.tex]{subfiles}
\begin{document}

\section{Tristan's Notes}

\begin{enumerate}
    \item 
We will show that: for $n\geq 2$ and $(n,a)=1$ that:\\
\begin{equation} 
    (X+a)^n=X^n+a\in (\mathbf{Z}\backslash n\mathbf{Z})[X]\iff n\in \mathbf{P}
\end{equation}
To proof this we will use two theorems which we state now, they are the binomial theorem and Fermat's little theorem:
\begin{itemize}
    \item For any prime we have $a^p=a\in (\mathbf{Z}\backslash p\mathbf{Z})$.
    \item For any natural number $n$ we have $(a+b)^n=\sum_{k=0}^n \binom{n}{k}a^kb^{n-k}$ these theorem can be generalised for any ring which is an exercise in DF. So then also in particular we can do them in polynomial rings.  
\end{itemize}
From this we can now compute the number $\binom{p}{k}=\frac{p!}{k!(p-k)!}$ noticing that $p$ is a prime we notice that $p$ is indivisible by any factors in the denominator thus: $\binom{p}{k}=pc$ for some $c$ and any $k$ in this sum except for two terms. This means that in the quotient $(\mathbf{Z}\backslash p\mathbf{Z})[X]$ this reduces to $0$ for any $k\neq 0$ or $k=p$. If $k=0$ or if $k=p$ then we see that we have $x^p+a^p$ so then using Fermat's little theorem for $a$ implies:
\begin{equation}
     (X+a)^n=X^n+a\in (\mathbf{Z}\backslash n\mathbf{Z})[X]
\end{equation}
From this we see that if we assume the number $p$ is prime we see that the equation hold now what remains to be shown is the other side of the inclusion. \\
For the other implication we use contra proposition, we assume $p$ is not a prime then it must have some factor $q^k$ by the FTA. Now we consider the coefficient of $X^q$ it has $\binom{n}{q}=\frac{n!}{(n-q)!q!}=\frac{n(n-1)...(n-q+1)}{q!}a^{n-q}$ since the numerator now contains $n$ it is divisible by $q^k$ but not divisible by $q^{k+1}$. The full thing then is not divisible by $q^k$ since we divide out $q$\\
We have now seen that $q^k\not|\binom{n}{q}$ now from here we consider $q^k$ and $a^{n-k}$ we assumed $(n,a)=1$ then from this and bezout's identity we see that:
\begin{align}
    ax+ny=1\\
    a^{n-k}(a^{k-n+1}x)+q^k(\Tilde{n}y)=1
\end{align}
this now means there are new $\Tilde{x}$ and $\Tilde{y}$ so that $a^{n-k}\Tilde{x}+q^k\Tilde{y}=1$ hence we notice again by the same identity that $a^{n-k}$ and $q^k$ are co-prime.\\
Now we can notice that if $n|\binom{n}{q}a^{n-q}$ and $q^k|n$ that then $q^k|\binom{n}{q}a^{n-q}$ this however is a contradiction thus we notice that $n\not|\binom{n}{q}a^{n-q}$ and therefore we notice there is a coefficient in the expansion of that is not the first or the last so we see that $(X+a)^n\neq X^n+a\in (\mathbf{Z}\backslash n\mathbf{Z})[X]$ which show the negation of the other side, it follows the other side of the inclusion holds.\\
\end{enumerate}


\begin{enumerate}

\item First step of the proof 
During each step we will primarily provide examples to what is said sometimes we additionally state results from other courses al these are from abstract algebra by dummit and foote since this is what we used for rings and fields. \\
We begin by stating what we want to proof the theorem is as followed: the aks algorithm produces a prime that is: Given steps 1-6 of the procedure the algorithm will produce a prime. \\ 
To do this  we suppose not that is we suppose there is some number $n$ which we input into $ALG(n)$ with as a result $ALG(n)$ is prime even though initially $n$ was a composite number.  More concretely this means that all assumptions of the steps are assumed true and $n$ is a composite we will use the other assumptions later. First notice that by the fifth step we have 
\begin{equation}
    (X+a)^n = X^n+a \in \mathbf{Z}[x]\backslash(n,X^r-1)
    \label{e1}
\end{equation}
Then if $n$ is a composite and by the FTA we notice there is a prime factor $p\neq n$ so that $pk=n$ this means in terms of $(\ref{e1})$ that for some polynomials $p_1(x),p_2(x)$:
\begin{equation}
    \begin{split}
        (X+a)^n &= X^n+a+pkp_1(x)+(X^r-1)p_2(x)\\
                &= X^n+a\in \mathbf{Z}[x]\backslash(p,X^r-1)
                \label{e2}
    \end{split}
\end{equation}
This now is the new equation. Since $\mathbf{Z}$ is a UFD we notice that $\mathbf{Z}[X]$ is a UFD which means that for any element in $\mathbf{Z}[X]$ there must be a irreducible factorisation of that element. More concretely this verifies that $X^r-1$ has such a factorisation. From here we notice that being a irreducible factor in $\mathbf{Z}[X]$ may not coincide with being one in $(\mathbf{Z}\backslash p\mathbf{Z})[X]$ as an example we show one such instance: For example take $X^2+3X-6$ then we notice that since $3|3$ and $3|-6$ yet $9$ doesn't divide the last coefficient we have by Eisenstein's theorem from section 9.4 that this is irreducible. However this polynomial no roots in $(\mathbf{Z}\backslash 5\mathbf{Z})[X]$ so that by another criteria from that section it is not irreducible over this polynomial ring since its degree $<=2$. From this we notice we require a polynomial which is irreducible in the ring $(\mathbf{Z}\backslash p\mathbf{Z})[X]$ we name this polynomial $h(x)$ and more or less the same argument as before verifies that:
\begin{equation}
    \begin{split}
        (X+a)^n &= X^n+a+pp_1(x)+h(x)k(x)p_2(x)\\
                &= X^n+a\in \mathbf{Z}[x]\backslash(p,h(x))
                \label{e3}
    \end{split}
\end{equation}
Were $h(x)|X^r-1$ since its a factor of the polynomial. From here we are first going to compute what this new ring is this requires a little result called the third isomorphism theorem (this was stated in group theory, but never used it is also known as the divide above and below by a fraction for ideals or normal subgroups in group theory): 
\begin{equation}
    \mathbf{Z}[X]\backslash(p,h(x))\cong (\mathbf{Z}\backslash p\mathbf{Z})[X]\backslash(h(x))
\end{equation}
\textbf{Proof} to show this we use the theorem stated we won't verify we can use it:
\begin{equation*}
    \mathbf{Z}[X]\backslash(p,h(x))\cong (\mathbf{Z}[X]\backslash (p))\backslash (p,h(x))\backslash (p)) 
\end{equation*}
Now we use two things firstly by section 9.1 we have that the top of the left side is isomorphic to $(\mathbf{Z}\backslash p \mathbf{Z})[X]$ this is a field of $p$ elements. The right side is isomorphic to $(h(x))$ which is a irreducible polynomial in this field of order $m=\deg{h(x)}$. Then by an exercise from section 9.2 we notice this iso is in fact a finite field of $p^m$ elements. If we take away its zero we notice it has $p^m-1$ elements last this field is cyclic by a corollary from section 9.5. Next we consider some elements of $(\ref{e2})$. Then we can notice these generate a subgroup and we have that $g\in G$  means $g$ is nonzero. If this were not so we would see $x+a\in G$ and $x+a=0$ meaning that $(x+a)^n=0^n=0$ however this is by the fifth step $x^n+a$ so that $x^n=-a=x$ by the fact that $x+a=0$. Now we notice that since $x(x^{n-1}-1)=0$ and since $x\neq 0$ that $x^{n-1}=1$. From this and  $x^{n-1}-1|x^r-1$  which we will verify later we have there is only
irreducible-element namely $x^{n-1}-1$ and thereby then $d=1$ which can't be because even the lb for $d$ is $\log(3)^2>1$ so we conclude all elements are nonzero. Next (to give an example) we consider some factorisation of $g(x)=\Pi_{0\leq a \leq A}(x+a)^{e_i}\in H$ we have that a exponent is 
\begin{equation}
g(x)^n = \Pi_a((x+a)^n)^{e_i}=\Pi_a(x^n+a)^{e_i} = g(x^n) \in \mod {p,x^r-1}
\end{equation}
Which shows $g$ is a multiplicative function for this number $n$ by the assumption. More generally we define $S$ to be the set of values $k$ were this function is multiplicative, we have seen $n\in S$ and since $p$ is a prime we also have that $p\in S$ by the original theorem.
\item Second step of the proof we will now obtain a upper bounds on the size of $G$ using the set $S$. We start with lemma 4.1 this is that the set of multiplicative functions $g(x)$ is closed: that is if $a,b\in S$ then $ab\in S$:\\
\textbf{Proof} We assume the stated facts that is there is some $k,l$ so that $g(x)^k=g(x^k)\mod{p,x^r-1}$ and the same for $k$: notice first that for any $x$ we have $g(x)^k=g(x^k)\mod{p,x^r-1}$ this then apparently also work for $x=x^l$, from which we deduce that $g(x^l)^k=g((x^l)^k)\mod{p,(x^{l}})^r-1$. And since we know that $x^r-1|(x^l)^r-1$ we then also notice that this is true in $x^r-1$ by again writing it out which was done in the first part (going from 4.1 to 4.2 and to 4.3.)\\
Now we can start to compute this we notice it is 
\begin{equation}
    g(x)^{ab}=(g(x)^a)^b=(g(x^a))^b=g(x^{ab})\mod{p,x^r-1}
\end{equation}
we have used first the definition of exponentiation in a group, second the fact that $a\in S$ third the fact that $b\in S$ and fourth the previously stated part. Notice the main issue in this proof is to verify the multiplication is again in the same group and not that it is true. 
We consider the second lemma that states that, this idea is used more often (it also verifies were I say $d=1$): 
given $x^{rk}-1$ we can notice it has a divisor $x^r-1$ by using $p(x)=1+x+x^2+...x^k$ as its multiplicative. This then shows that:
\begin{equation}
        x^{a-b}-1=(x^r-1)p(x)
\end{equation}
so that $x^r-1$ divides this, less difficult is to notice that: $x^b(x^{a-b}-1)=x^a-x^b$ which shows the second argument made. The next argument made follows from the first observation that for $a,b\in S$ $v-u|g(v)-g(u)$ this is true because the by it being the same in the polynomial-ring, then noting that $g(x^a)=g(v)$ and that also for $x^b$ so then $x^a-x^b=u-v|g(u)-g(v)$. Noting again that $a,b\in S$ we can see that $[g(x)^a]=[g(x^a)]=[g(x^b)]=[g(x)^b]$ were the middle equality follows from the fact that $x^a-x^b|g(x^a)-g(x^b)$ since this means that 
\begin{equation}
    (x^a-x^b)k(x)+g(x^b)=g(x^a).
\end{equation}
The second function is now remainder of ideal $(p,x^r-1)$ and therefore these are in the same class.  Notice the group is multiplicative so that we can divide left side by the right and notice that $g(x)^{a-b}=1$. On the other hand we notice $G$ can be generated by  a single element so that $g(x)^{a-b}=1=g(x)^{k|G|}$ which is what was to be shown this $k|G|$ comes from Lagrange's theorem. \\
Next Since $n$ is not a perfect power we have that  $n^i\neq p^j $  this next implies that all $n^ip^j$ are distinct, this is more precisely: $i\neq I$ or $j\neq J$ then $n^ip^j\neq n^Ip^J$ to show this we assume the opposite that is $n^ip^j=n^Ip^J$ then if $i=I$ and $j\neq J$ then by the second we have that WLOG $p^j>p^J$ so that then clearly $n^ip^j>n^Ip^J$ which is against the original assumption. If we assume that $i>I$ then this argument work much the same so we see all $n^ip^j$ are distinct when $i$ or $j$ are. \luc{Next we consider the cardinality of these unique $n^ip^j$ this is by the product formula for cardinality: $|n^ip^j|=|n^i||p^j|=(\sqrt{|R|}+1)(\sqrt{|R|}+1)>|R|$ and then from this we see that if we consider the classes of $R$ we must have that two classes of $n^ip^j$ overlap by the pigeonhole principle. Thus  $[n^ip^j]=[n^Ip^J]\in \mod{R}$. Recalling the proven lemmas we thus see that: 
\begin{equation}
    \begin{split}
        |G|\leq k|G|&=|n^ip^j-n^Ip^J|\\
        &\leq (np)^{\sqrt{|R|}}-1\\
        &<n^{2{\sqrt{|R|}}}-1
    \end{split}
\end{equation}
The first part is true because of the previous remark and lemma $4.2$. Next we notice that if $i,j\leq \sqrt{|R|}$ then therefore $n^ip^j\leq (np)^{\sqrt{|R|}}$. Also we have $n^Ip^J\geq 1$ so that $-n^Ip^J\leq -1$ hence in total this leads to the second inequality. The last inequality follows from the fact that if $p|n$ then $p<n$.  From here we verify a stricter upper bound namely: $|G|\leq n^{\sqrt{|R|}}-1$ for this we will show that$\frac{n}{p}\in S$(showing this is really lengthy and doesn't add to the proof).}
\item Final step of the proof: 
We have some more steps first we can consider $[f(x)]=[g(x)] \in \mod{p,h(x)}$ this implies that 
\begin{equation}
    \begin{split}
        f(x)      &= g(x) + pp_1(x)+h(x)p_2(x)\\
        f(x)-g(x) &= pp_1(x)+h(x)k(x)\\
        f(x)-g(x) &= h(x)k(x) \in \mod{p}.
    \end{split}
\end{equation}
Next we can notice that by the degree formula and upper bound on the degree of $f,g$ that $deg(f(x)-g(x))<\deg h(x)$ however we notice that $\deg(h(x)k(x))=\deg(h(x))+\deg(k(x))$ which can only happen if $k(x)=0\in \mod{p}$. From here then notice that $[f(x)]=[g(x)]\in \mod{p}$ this means that if $f,g$ are the same then reduction will not leave this unaltered, this is what we will additionally do for polynomials that are additionally in the same class in $G$. The goal of this is to show that $f,g $ in the same class are not counted as distinct elements. Now we will show this is the case:
We assume that $f,g\in \mathbf{Z}[X]$ and $[f],[g]\in G$ then we have for $f,g$ as degree less then $|R|$ that $[f(x)]=[g(x)]\in \mod{p}$
\textbf{Proof} We define $\delta(y) = f(y)-g(y)$ for any $k\in S$ we then have that: $\delta(x^k)= f(x^k)-g(x^k)=f(x)^k-g(x)^k=0\in \mod{p,h(x)}$ were the first line follows from the definition the second because $k\in S$ and the last by the assumption that $[g(x)]=[f(x)]\iff [g(x)^k]=[f(x)^k] $ now if $x\in \mathbf{F}_p$ were this is the field generated in the beginning then since $\mathbf{F}$ is iso to some cyclic group we have that $x$ has order $r$ and we have that $x^k$ for $k\in R$ are all distinct. From here we next have that $\delta$ has more then $|R|$ roots at the same time $\deg{f(x)}<|R|$ which implies by another theorem from 9.5 that the amount of roots of $f$ can be at most $|R|$ these things can only happen if $\delta(y) = 0$ this is allowed by a proposition from 9.5(the degree argument of this). This next means that $f,g$ share class in $\mod{p}$.
\luc{Here we now use this we define $B$ to be the floor of $2\log{n}$ then we can notice that first $|R|=\sqrt{|R|}\sqrt{|R|}>\sqrt{|R|}\log{n}=B$ were we assume that $|R|>\log{n}^2$ which is one of the assumptions in the algorithm. From here we now know that each $\Pi_{a\in T}(x+a)$ gives a truly distinct representation by lemma 4.3. To better verify this notice that this lemma can be restated as if $f(x)$ and $g(x)$ have distinct classes in $\mod{p}$ then they will have distinct classes in $\mod{p,h(x)}$ thus when we count the elements in $G$ and form all the subsets of $\{0,..,B\}$ we know non of these will be double counted moreover we see since $|R|>B$ that there are element of the initial group so that we don't count them with this product. We thus see that  $|G|\ge 2^{B+1}-1$ were we now have that $2^{B+1}-1=2\cdot n^{\sqrt{|R|}}-1>n^{\sqrt{|R|}}-1$. This is contrary to the previous upper bound hence we reach a contradiction. Here we have used that since $B=2\log(n)$ so that $2^B=n$.}
\item We next consider the AKS algorithm, given the above proof we have actually verified that the following steps produce a prime number:
\begin{enumerate}
    \item We find out if $n$ can be written as $n=a^b$ for $a,b>1$ if this can be  done we notice $n=a(a^{b-1})$ hence $n$ is a composite number. 
    \item Find the smallest $r$ so that the order of $n\mod{r}>(2\log{n})^2$ why $r$ has to be small is important latter. That this order has to be larger then the right side is to satisfy the lower bound on $|G|$
    \item Check for  $a\in [2,\min{r,n-1}]$ if $a|n$ if this happens output composite. This follows from a primality test for numbers 
    \item If $n\leq r$ output prime 
    \item for $a\in [1,\sqrt{\phi(t)}2\log{n}]$
    check if congruence $(X+a)^n= X^n+a\in \mod{X^r-1,n}$ is true if not output composite. This bound is to garantee all elements that are generated are distinct. 
    \item Ouput prime 
\end{enumerate}
Notice this procedure is a sieving method at each stage we sieve out non prime numbers until the last stage were we apply the proven result. The reason we do this is because it might still be expensive to try and solve the last equation in one go. The code for this procedure is given below:
To informally show our method is correct we consider the first 10000 numbers and check them for primality. With we see this agrees with what is given in sage:


Last we verify gauss his prime number theorem, for this we consider the primes up until $104729$ We count the amount of primes the algorithm gives us this is: $k = 10^4$ dividing this by the amount of numbers we have counted gives us  To compare this with the theorem we compute $\frac{N}{\log{N}}=\frac{}{}$ this comes close to the actual value so we state the theorem is correct for this amount of numbers.  

\end{enumerate}

\end{document}