\subsection{Challenges}


- Only one week to get acquainted with Lean

- Lean is very much still in development, and lacks accessible documentation and examples for beginners. Lack of tutorials besides the 'Mathematics in Lean' one.

- Certain constructions are obscure to the neophyte

- Documentation is sometime difficult to search. Even once acquainted with Lean naming conventions and hierarchy, finding the proper method is non trivial.
Absence of NL description of most theorems, making searching for specific methods trickier than needed. In any instance, we had to guess the naming used for the methods we were looking for. 

Often, finding how to tell Lean to do what we wanted it to do was quite hazardous. This was especially tricky when  we had to define certain subgoals. Classical (pen and paper) proof writing allow to go from one argument to the other or to easily manipulate constructions in ways that is not trivial to do in Lean.



- Typing: we often simply state: Let x be an integer, let g be a group.
However, when defining our hypothesis and variables in Lean we found ourselves questioning whether this was the most useful way to define it in. For example, there are more functions defined for Naturals than integers, and since we we're looking at positive integers this is equal.

For groups, there are Monoids, CommSemiring, DivisionSemiring.toSemiring 

- Despite being familiar with most concepts before, when formalizing one is forced to really form a more concrete idea of what every definition exactly means, breaking it down to the smallest definition or hoping you will find and understand an already made function in the Mathlib library 