\documentclass[../main.tex]{subfiles}

\begin{document}

\section{Notes}

% \textbf{Theorem}: (please proof read it and correct whatever u see)\\\\
The if and only if statement can be rewritten as:
\begin{equation*}
    \text{$n$ is prime}\quad\Longleftrightarrow\quad(x + a)^n - (x^n + a)\equiv 0 \Mod n
\end{equation*}
Which essentially says that $n$ is prime if and only if $n$ divides all coefficients of the polynomial $(x+a)^n-(x^n+a)$. By the binomial theorem, we can write this polynomial as:
\begin{equation}
    \begin{split}    
        \left(\sum_{k=0}^{n} \binom{n}{k}x^{n-k}a^k\right)-\left(x^n+a \right)
        & =\left(\sum_{k=1}^{n-1} \binom{n}{k}x^{n-k}a^k\right)+ x^n +a^n-\left(x^n+a\right)
        \\ &= \left(\sum_{k=1}^{n-1} \binom{n}{k}x^{n-k}a^k\right) + \left(a^n-a\right)
    \end{split}
    \label{binom}
\end{equation}
In order to prove that (\ref{binom}) can be divided by $n$, or equivalently, it is congruent to $0 \Mod n$, we use the two following statements:
\begin{itemize}
    \item \textbf{Claim}: $n$ is prime if and only if $\binom{n}{k}\equiv 0 \Mod n$ for $k\in\{2,3,4,\dots,n-1\}$. 
    \item \textbf{Fermat's little theorem}: If $n$ is a prime number, for any integer $a$,
    \begin{equation*}
        a^n \equiv a \Mod n \quad\text{or,}\quad a^n -a \equiv 0 \Mod n.
    \end{equation*}
\end{itemize}
Remember that we have to show in both directions.
\begin{equation*}
    \text{$n$ is prime}\quad\Longleftrightarrow\quad \left(\sum_{k=1}^{n-1} \binom{n}{k}x^{n-k}a^k\right) + (a^n-a)\equiv 0 \Mod n
\end{equation*}

\begin{itemize}
    \item $\Longrightarrow$. Assume $n$ is prime. Then $\sum_{k=1}^{n-1} \binom{n}{k}x^{n-k}a^k\equiv0\Mod n$ by our claim, and $(a^n-a)\equiv 0 \Mod n$ by Fermat's little theorem. 
    \item $\Longleftarrow$. Assume $\left(\sum_{k=1}^{n-1} \binom{n}{k}x^{n-k}a^k\right) + (a^n-a)\equiv 0 \Mod n$. Then it must hold that $\sum_{k=1}^{n-1} \binom{n}{k}x^{n-k}a^k\equiv 0 \Mod n$ which implies that $n$ is prime by our claim.
\end{itemize}

\section{Theorem of Agrawal, Kayal, and Saxena}

\begin{theorem}[Agrawal, Kayal and Saxena] \label{aks}

Given an integer $n \geq 2$, let $0 \leq r < n$ for which $\mbox{ord}_{r}(n) > (\log{n})^{2} \Mod r$. Then $n$ is prime if-and-only-if
\begin{itemize}
 \item $n$ is not a perfect power,
 \item $n$ does not have any prime factor $\leq r$,
 \item $(X + a)^{n} \equiv X^{n} + a \bmod (n, X^{r} - 1)\; \forall a \in \mathbb{Z}$ such that $1 \leq a \leq \sqrt{r} \log{n}$.
\end{itemize}
\end{theorem}

\subsection{Proof overview}
The goal of this section is to give an outline of the correctness proof of theorem \ref{aks}. We will not focus on a rigorous step by step proof as we are mostly interested in implementing it in Lean, however we still need to be able to understand it to use the proof-assistant.  
\\\\
It is important to first make sense of the following theorem:
\begin{theorem} \label{theorem:1}
Given an integer $n \geq 2$, for all integers $a$ coprime to $n$, $n$ is prime if and only if

\begin{equation*}
    (X + a)^{n} \equiv X^{n} + a \Mod n
\end{equation*}

\end{theorem}
The proof of the theorem above revolves around the binomial theorem, Fermat's little theorem and the following claim: $n$ is prime if-and-only-if $\binom{n}{k} \equiv 0 \Mod n$ for $k \in \{ 2,3,4,...,n-1 \}$.\\\\
The forward direction of the if and only if statement of the AKS is pretty straightforward once we assume theorem \ref{theorem:1}. If we let $n$ be prime, the first two conditions in theorem \ref{aks} follow immediately, and the congruence relation holds because every polynomial $\Mod {p,x^r-1}$ is also $\Mod p$.
\\\\
The reverse direction requires some more elaborate mathematics. We will prove the implication by contradiction, assuming that $n$ is composite and that all conditions of theorem \ref{aks} hold. Let $p$ be a prime number dividing $n$ and $h(x)\in (\mathbb{Z/p\mathbb{Z})[x]}$ an irreducible factor of $x^r-1$. It then follows that:
\begin{equation}
\label{red}
    (X + a)^{n} \equiv X^{n} + a \Mod {n,x^r-1} \Longrightarrow (X + a)^{n} \equiv X^{n} + a \Mod {p,h(x)}
\end{equation}
Note that we are now dealing with the quotient ring $\mathbb{F}:\equiv(\mathbb{Z}[x]/(p,h(x))$ where the ideal $(p,h(x))$ is irreducible. Hence, it is isomorphic to a field, from which we can take the cyclic multiplicative group $\mathbb{F}\backslash \{0\}$, this will be important for the construction of some groups ahead. The reduction in (\ref{red}) has the purpose of allowing us to deal with a field, where the congruence are easier to work with.
\\\\
To continue with the prove we have to define a few sets first:
\begin{itemize}
    \item Let $H$ be the elements $\Mod {p,x^r-1}$ generated multiplicatively by $x,x+1,...,x+\lfloor A \rfloor $
    \item Let $G$ be the reduction of $H\Mod {p,h(x)}$ 
    \item Let $S$ be set of positive integers $k$ for which $g(x^k) \equiv g(x)^k \Mod {p,x^r-1}$ for all \\$g(x) := \prod_{0\leq a\leq A}(x+a)^{e_a}\in H$ where $e_a$ is an arbitrary exponent dependent on $a$.
    \item Let $R$ be the subgroup of $(\mathbb{Z}/r\mathbb{Z})^*$ generated by $n$ and $p$. 
\end{itemize}
We now will find contradicting bounds on $|G|$.
\subsubsection*{Upper bound}
Consider the integers of the form $n^ip^j$ with $0\leq i,j\leq \sqrt{|R|}$. Since $n$ is not a power of $p$, they are distinct. It also holds that there are more integers $n^ip^j$ than $|R|$ so two must be congruent, namely, $n^ip^j\equiv n^Ip^j \Mod {r}$ for some other combination $0\leq I,J\leq \sqrt{|R|}$. We now use two lemmas from \ref{}. The first one states that for all $a,b\in S$, we have that $ab\in S$. The second says that if $a,b\in S$ and $a\equiv b \Mod r$, then $a\equiv b \Mod {|G|}$. It then follows that:
\begin{equation*}
    |G|\leq |n^ip^j-n^Ip^J|\leq (np)^{\sqrt{|R|}}-1\leq n^{2\sqrt{|R|}}-1
\end{equation*}
According to Andrew Granville, $n/p$ is also in $S$, and can be substitute $n$ in the process above, to give a smaller upper boundary: $|G|\leq n^{\sqrt{|R|}}$.
\subsubsection*{Lower bound}
To give a lower bound on $|G|$ we must show that there are many distinct element in $G$.\\\\
Remember that $R$ was generated multiplicatively by $n$ and $p$ in $(\mathbb{Z}/r\mathbb{Z})$, hence it contains all elements of the form $n \Mod r$. By assumption, the order of $n\Mod r$ is greater than $(\log n)^2$, so $R$ has at least more than $(\log n)^2$ elements. We can now say that $|R|>B$ where $B=\lfloor \sqrt{|R|}\log n\rfloor$.\\
Let $T$ be a proper subset of $\{0,1,2,...,B\}$. So, $|T|\leq B<|R|$. If we consider the polynomials
\begin{equation}
\label{pol}
    \prod_{a\in T}(x+a)
\end{equation}
we realize that they are elements of $G$ and their degree is the number of elements $a\in T$, namely, $|T|$ which is less than $|R|$, this will be important to invoke the lemma from \ref{} below.
\begin{lemma}
\label{lemma:4.3}
    Suppose that $f(x), g(x)\in\mathbb{Z}[x]$ with $f(x)\equiv g(x) \Mod {p,h(x)}$ such that their reductions in $\mathbb{F}$ are in $G$.
    If $deg(f(x)),deg(g(x))<|R|$, then $f(x)\equiv g(x) \Mod p$
\end{lemma}
By taking the contrapositive of lemma \ref{lemma:4.3}, it follows that the polynomials \ref{pol} are distinct in $G$ if they are distinct in $(\mathbb{Z}/p\mathbb{Z})[x]$ for every distinct $T$.
Hence we just need to count the possible sets $T$ which is $2^{B+1}-1$.
So we can write: $|G|>2^{B+1}-1$. We now just have to show that $2^{B+1}-1>n^{\sqrt{|R|}}-1$:
\begin{align*}
& 2^{B+1}-1>n^{\sqrt{|R|}}-1 \Longleftrightarrow 2^{B+1}>n^{\sqrt{|R|}} \Longleftrightarrow \log {(2^{B+1})} > \log {(n^{\sqrt{|R|}})}\Longleftrightarrow B+1 > \sqrt{|R|}\log n\\
& \Longleftrightarrow \lfloor\sqrt{|R|}\log n\rfloor+1 > \sqrt{|R|}\log n
\end{align*}
Hence we proved that $|G|>n^{\sqrt{|R|}}-1$, contradicting the upper bound of $|G|$ we got before.

\newpage


\end{document}