\documentclass[../main.tex]{subfiles}

\begin{document}

\begin{theorem}[Fermat's Little Theorem] \label{theorem: flt}
    If $n$ is prime, then $n$ divides $a^{n} - a$ for all integers $a$.
\end{theorem}

\begin{theorem} \label{theorem:1}
Given an integer $n \geq 2$, for all integers $a$ coprime to $n$, $n$ is prime if and only if

\begin{equation}
    (X + a)^{n} \equiv X^{n} + a \Mod n.
\end{equation}

\end{theorem}

\begin{proof}

We begin by rewriting $(X + a)^{n}$ using the Binomial theorem:

\begin{equation*}
    \begin{split}
        (X + a)^{n} &= \sum_{k=0}^{n} \binom{n}{k} X^{n-k} a^{k} \\
        &= \sum_{k=0}^{n} \frac{n!}{(n-k)!\;k!} X^{n-k} a^{k} \\
        &= \frac{n!}{(n-0)!\;0!} X^{n} + \sum_{k=1}^{n-1} \frac{n!}{(n-k)!\;k!} X^{n-k} a^{k} + \frac{n!}{(n-n)!\;n!} a^{n} \\
        &= X^{n} + \sum_{k=1}^{n-1} \binom{n}{k} X^{n-k} a^{k} + a^{n}.
    \end{split}
\end{equation*}

To proceed with the proof, we utilise the following result which we prove in the appendix:

\begin{lemma} \label{lemma:1}
    $n$ is prime if-and-only-if $\binom{n}{k} \equiv 0 \Mod n$ for $k \in \{ 2,3,4,...,n-1 \}$.
\end{lemma}

Continuing, we consider two directions:

$(\Longrightarrow)$: Assume that $n$ is prime. Then $a^{n} \equiv a \Mod{n}$ by the Fermat's little theorem. By Lemma \ref{lemma:1}, we have $\sum_{k=1}^{n-1} \binom{n}{k} X^{n-k} a^{k} \equiv 0 \Mod{n}$. And so it follows that $(X + a)^{n} \equiv X^{n} + a \Mod n$.
\newline

$(\Longleftarrow)$: Assume now that $(X + a)^{n} \equiv X^{n} + a \Mod n$. We can rewrite it as $$ \Biggl( \sum_{k=1}^{n-1} \binom{n}{k} X^{n-k} a^{k} \Biggr) + (a^{n} - a) \equiv 0 \Mod n $$ using the expansion above. Then it must hold that $\sum_{k=1}^{n-1} \binom{n}{k} X^{n-k} a^{k} \equiv 0 \Mod n$ \colorbox{red}{which implies} that $n$ is prime by our claim.

\end{proof}

\begin{theorem}[Agrawal, Kayal and Saxena] \label{aks}

Given an integer $n \geq 2$, let $0 \leq r < n$ for which $\mbox{ord}_{r}(n) > (\log{n})^{2} \Mod r$. Then $n$ is prime if-and-only-if
\begin{itemize}
 \item $n$ is not a perfect power,
 \item $n$ does not have any prime factor $\leq r$,
 \item $(X + a)^{n} \equiv X^{n} + a \bmod (n, X^{r} - 1)\; \forall a \in \mathbb{Z}$ such that $1 \leq a \leq \sqrt{r} \log{n}$.
\end{itemize}
\end{theorem}

\end{document}